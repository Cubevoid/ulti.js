\documentclass[letterpaper,11pt]{article}
\usepackage[T1]{fontenc}
\usepackage[utf8]{inputenc}
\usepackage{lmodern}
\usepackage{amsmath}
\usepackage{amsfonts}
\usepackage{amssymb}
\usepackage{amsthm}
\usepackage{graphicx}
\usepackage{color}
\usepackage{xcolor}
\usepackage{url}
\usepackage{textcomp}
\usepackage{hyperref}
\usepackage{parskip}
\usepackage{geometry}
\geometry{letterpaper, portrait, margin=0.75in}

\title{Ulti.js Alpha Release}
\author{Colin De Vlieghere}
\date{\today}

\begin{document}

\maketitle

\paragraph{Description}
The main purpose of this library is to
help with visualizing and learning different team plays in the game of Ultimate Frisbee.
Ulti.js will allow for viewing some common preset team configurations, such as the 
``vertical stack'' or ``horizontal stack'', as well as any custom arrangement of players
on both teams and the disc, by the developer or by the user.
Some use cases for Ulti.js would be, for example, a coach using it in a ``whiteboard''
style web app with annotations to teach players about the game interactively. Another use case 
is someone writing a blog or lesson about the game, or as a reporting tool to show what happened 
in a real game, for example on a news site.

\paragraph{Implemented Features}
Features that are currently implemented include creating a new board representing a field,
vertical and horizontal field arrangements, the addition and removal of players, getting and setting a
player's location, jersey number, name, and type (offense/defense), showing and hiding the
disc, giving ownership of the disc to a player, and getting and setting the disc's position.
There is also a preset for the ``Vert Stack'', a common offensive formation in the game.
Finally, there are tooltips that show a player's name and jersey number that appear on hover,
although they can also be manually toggled on and off.

\paragraph{Deployed web page}
\href{https://fathomless-tor-21034.herokuapp.com/examples.html}{https://fathomless-tor-21034.herokuapp.com/examples.html}

There are three examples of the library's usage on the page.
The first one shows a default horizontal field with one player and the disc moving to a different
position after one second.
The second one shows the Vert Stack preset with both offense and defense players. It also includes
some buttons to manipulate the disc's visibility and owner.
Finally, the last example shows an initially blank board, with a button that will populate the
Vert Stack preset. There is also a button to clear the board and reset it to default.
This example also includes buttons to show and hide player tooltips manually.

\paragraph{JavaScript objects}
There are three JS objects that are used in ulti.js. The first is the main \texttt{UltiBoard} class, and the
other two are the \texttt{Player} and \texttt{Disc} objects that are stored within an \texttt{UltiBoard}.

Here is an example of an instance of \texttt{UltiBoard}, not including its functions (which make up
the library's main API):

\begin{verbatim}
  {
    "debug": true,
    "rootElement": <div id="ulti-root" style="width: 1100px; height: 400px;">,
    "players": [
      {
      "id": 1,
      "x": 550,
      "y": 200,
      "name": "Colin De Vlieghere",
      "number": 23,
      "type": "offense",
      "rotation": 0,
      "tooltip": <div class="ulti-tooltip">
      }
    ],
    "_playerSizePx": 20,
    "disc": {
      "x": 50,
      "y": 50,
      "size": 20,
      "owner": {
        "id": 1,
        "x": 550,
        "y": 200,
        "name": "Colin De Vlieghere",
        "number": 23,
        "type": "offense",
        "rotation": 0,
        "tooltip": <div class="ulti-tooltip">
      }
    },
    "discShown": true,
    "fieldWidth": 400,
    "fieldLength": 1100,
    "vert": false,
    "_playerId": 2
  }
\end{verbatim}

\begin{itemize}
  \item In this example, the \texttt{debug} flag is set, meaning the code will have verbose outputs in the console.
  \item \texttt{rootElement} is just there for convenience, and keeps a reference to the board in the DOM.
  \item \texttt{players} is an array of \texttt{Player} objects, with coordinates, rotation, and type corresponding to 
        the element in the DOM (corrected for the player's size, \texttt{\_playerSizePx}).
        The Player object also has an id corresponding to the CSS id, and a reference to its tooltip in the DOM.
  \item \texttt{disc} is an object representing the disc in the DOM. It too has coordinates and a size in pixels,
        as well as an \texttt{owner} which is a reference to a Player object.
  \item \texttt{discShown} is a flag representing whether or not the disc is shown in the DOM.
  \item \texttt{fieldWidth} and \texttt{fieldLength} represent the dimensions of the field in the DOM (in pixels),
        corrected for orientation (i.e. when the field is horizontal \texttt{fieldWidth} corresponds to CSS height
        and \texttt{fieldLength} corresponds to CSS width).
  \item \texttt{vert} is simply a flag describing the field's orientation (vertical or horizontal).
\end{itemize}

\newpage
\paragraph{API Functions}
In addition to the \texttt{constructor} for the \texttt{UltiBoard} class, here are the main API functions:

\begin{itemize}
  \item \texttt{clear()}: clears the board of players and the disc
  \item \texttt{addPlayer(x, y, name, jerseyNumber, type, rotation)}: adds a new player onto the field
  \item \texttt{updatePlayers()}: updates the players in the DOM from the \texttt{players} array 
                                  (useful after removing or modifying a player)
  \item \texttt{showDisc()}: shows the disc
  \item \texttt{hideDisc()}: hides the disc
  \item \texttt{setDiscPosition(x, y)}: sets the disc's position relative to its owner, or to the field if there is no owner
  \item \texttt{setDiscOwner(newOwner)}: sets the disc's owner to a player
  \item \texttt{showNames()}: manually shows the tooltips with player numbers and names
  \item \texttt{hideNames()}: manually hides the tooltips with player numbers and names
  \item \texttt{vertStack()}: Vertical Stack preset
\end{itemize}

\paragraph{Next Steps}
I want to implement the Horizontal Stack preset next, as well as the ability for the user to drag and rotate 
players and the disc around. The latter should be somewhat challenging, since it involves more complex mouse 
actions and moving around several DOM elements. Also, in every function I create, I need to account for both
the vertical and horizontal field orientations, which adds some complexity to the implementation.

\end{document}